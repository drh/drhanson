%
%	This is the sample paper marked up with the LaTeX SPE macros.
%	Note that references are prepared using BiBTeX, and are in the
%	BiBTeX bibliography "spe.bib"
%
\documentstyle[times,spe,epsf]{article}
\begin{document}

% comment out the following line to include time and date in footer
\def\sysdetails{}

\title{Sample Paper for SPE}
\author{A. N. OTHER}
\affiliation{Department of Computation,
The Institute of Science and Technology,
P.O. Box 88,
Chichester PO19 1UD, UK}

\titlehead{SAMPLE PAPER FOR SPE}
\authorhead{A. N. OTHER}
\received{1 March 1988}
\revised{25 March 1988}
% volume issue month year first-page last-page
\spe{1}{1}{JANUARY}{88}{1}{4}

\maketitle

\author{J. OTHER}
\affiliation{Department of Typography,
The Institute of Science and Technology,
Science Park, Main Road,
Chichester, West Sussex, PO19 1UD, UK}
\makeauthor

\begin{abstract}
The paper should start with a summary (maximum length 200 words) of the entire
paper, not of the conclusions alone. It
should be set in nine point bold on a baseline of ten points. The
heading should be in nine point bold capitals.
\end{abstract}

\bibliographystyle{spe}

\keywords{Data structures\quad Data types\quad Persistent data\quad
          Abstract data store\quad User interfaces}

\section*{KEY WORDS}

The key words (of which there should be no more than six) which follow the
summary should be set in
eight point medium. The words KEY WORDS should be in small capitals in
eight point medium. One line of white space should separate the
end of the summary and the key words.

\section*{GENERAL SPECIFICATIONS}

All papers submitted on disk will be set in Times Roman when
published in the journal. If you are unable to use this type face
your source file will be amended. The source file will also be
amended to incorporate any changes that may be required at the
later stages. In the light of these facts we would ask you not to
make any attempt to balance the pages in the file which is
submitted for publication.

The papers will be set to a width of 34 picas and a depth of 48
lines of body text. Short Communications will be set in two columns of
16$1\over2$ picas width. Running headline details are
controlled by the macros. If you are submitting CRC these will be
added prior to publication when the pagination is known.

The text should be set in eleven point medium on a twelve point baseline.
For Short Communications the text should be set in nine point medium on a ten
point baseline.

\section*{DETAILS FOR PAPER OPENING}

The title of the paper should be set in 18 point medium, centred.
The first letters of the main words should be capitalised.
One line of space should be set under the title.

For a Short Communication the words `Short Communication' should be set as the
title, with the title of the communication itself set in eleven point capitals
at the head of the first column.

The author's name should be set in eleven point
medium small capitals on a twelve point baseline.

The address details should be set in nine point italics on a
ten point baseline in the style shown above, without punctuation
at the end of lines.

\section*{FIRST LEVEL HEADING}

A first level heading should be set centred in eleven
point (nine point for a Short Communication) medium capitals with one and a
half lines of space above and half a line of space below.\cite{FastCom}
The first word of the first paragraph should not be indented.
Subsequent paragraphs should be indented one em of set.

\subsection*{Second level headings}

A second level heading should be set flush left in eleven
point (nine point for a Short Communication) bold with only the first word
capitalised. As with first
level headings one and a half lines of space should be set above
and half a line of space should be set below the heading. The
start of the first paragraph should be indented,
as should subsequent paragraphs, by one em.

\subsubsection*{Third level headings}

A third level heading should be set flush left in eleven
point (nine point for a Short Communication) italic with only the first word
capitalised. As with first
level headings one and a half lines of space should be set above
and half a line of space should be set below the heading. The
start of the first paragraph\cite{UFFS}
should be indented, as should subsequent paragraphs, by one em.

\subsubsubsection{Fourth level headings}
A fourth level heading should be indented by one em and set in eleven point
Times italic on a baseline of twelve points (for a Short Communication nine
points on a ten point baseline), with only the first word
capitalised. There should be half a
line of space above the heading and the text should run on from the heading
after a full point.

\section*{LISTS}

Lists should be numbered or
identified with letters. In each case half a line of white space
should separate the list from the preceding and subsequent text.

A list should be indented by one em. If the identifier varies in
width it should be aligned on the right.

The start of the list text should be separated from the
identifier by one en space.

\begin{enumerate}

\item
This is a numbered list which should show the
indentation level which is expected.

\item
List entries which extend to more than one line should
have second and subsequent lines aligned with the first text character.

\end{enumerate}

\section*{EXTRACTS}

\begin{quote}
An extract should be set in eleven point medium on a
baseline of twelve points (nine point on a ten point baseline for Short
Communications). The text should be indented by two ems on
each side and should be set with a line of white space
separating the extract from the preceding and subsequent text.
\end{quote}

\section*{FOOTNOTES}

Footnotes should be set in eight point medium on a
baseline of nine points. They should be indicated by symbols in the order
* \dag \ddag \S \P $\parallel$ on each page.
\footnote{
All footnotes should be set to the full measure (the column width for a Short
Communication) with a four pica length rule separating it from the text on
that page.} 

\section*{FIGURES}

If you are able to incorporate the figures in the
source file they should appear as close to the first reference as
possible. One line of space should separate the figure from the
preceding and subsequent text. If you are submitting the
illustration\cite{SPE}
as hard copy leave a nominal space in the file
and set the caption at this point.

\begin{figure*}[ht]
\begin{center}
\leavevmode
\epsfxsize=27.5pc
\epsffile{fig1.eps}
\end{center}
\caption{Illustration for sample paper. Reproduced from G. Santucci and P. A.
Sottile, `Query by diagram: a visual experiment for querying databases',
Software---Practice and Experience, Vol. 23, 317--340 (1993)}
\end{figure*}

The figure caption should be set in nine point italic on a ten
point baseline (eight point on a nine point baseline for Short
Communications). It should be set to the width of the illustration
and should be centred on the full measure. One line of space
should separate the caption from the illustration itself. The final full point
should be omitted from figure and table captions.

\section*{TABLES}

Tables should be set in nine point medium on a baseline
of ten points (eight point on a nine point baseline for Short
Communications). Vertical lines should not be set. Tables should be numbered
with roman numerals.

\begin{table*}[ht]
\begin{center}

\caption{Basic performance measures$^*$}
\begin{tabular}{@{} l l r @{}}
\tmedline
Operation & & \multicolumn{1}{r}{Cost} \\
\hline
Null kernel call & & 1 ms \\
Create/destroy link & & 3 ms \\
Disk block read or write (512 bytes) & & 110 ms \\
 & Local & Remote \\
 \cline{2-3}
Send/receive 0-byte message & 10 ms & 63 ms \\
Send/receive 1024-byte message & 25 ms & 280 ms \\
\tmedline
{\tabnote $^*$ A table footnote set in eight point medium.}

\end{tabular}
\end{center}
\end{table*}

\acknowledgements

If you are including any acknowledgements the heading
should be set in small capitals in eleven point medium (nine point for a Short
Communication). One and a half
lines of space should be set above and half a line of space should
be set below. The start of the first paragraph should not be
indented. The acknowledgements should appear at the end of the main text
before any appendices and the references.

\appendix
\section*{COMPUTER PROGRAMS}

Appendices should normally appear after the acknowledgements but before the
references.

Computer programs and program words in running text should be set
in 11 point Univers light (nine point for a Short
Communication). The use of bolder type for `keywords' should be
avoided unless, as in the case of Algol 68, it is essential, in which case 11
point Univers medium (eight point for Short Communications) should be used.

\begin{figure}[h]
\sf\obeylines
main() \{
\quad printf("This is an example program.$\backslash$n");
\}
\end{figure}

\endappendix

\section*{REFERENCES}

If you are including any references the heading should
be set in capitals in ten point medium. One and a half lines of
space should be set above and half a line of space should be set
below. The reference should be identified within the paper with a
sequential number set as a superscript.\cite{FastCom,UFFS,SPE}

The text of the references is to be set in nine point medium on a ten point
baseline (eight point on a nine point baseline for Short Communications). The
details of style depend on the nature of the reference. Some examples are
given below.

\bibliography{sample}

\end{document}
